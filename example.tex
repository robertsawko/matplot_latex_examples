\documentclass[a4paper,12pt]{article}

\date{\today} 

\usepackage{amsmath}
\usepackage{amsfonts}
\usepackage{authblk}
\usepackage{booktabs}
\usepackage{fontspec}
\usepackage{graphicx}
\usepackage{hyperref}
\usepackage{pgf}

%\addbibresource{references.bib}
%Get rid of these nasty lines around links
\hypersetup{pdfborder=0 0 0} 

\author{Robert Sawko}
\affil{Department of Engineering Computing, Cranfield University}
\title{Examples with Python PGF support}

%Setting a nicer(?) font (which need xetex)
\defaultfontfeatures{Scale=MatchLowercase,Mapping=tex-text}
\setmainfont{Liberation Serif}

\begin{document}
\maketitle

\begin{abstract}

  This document presents examples usage of \texttt{matplotlib} together with
  \LaTeX. We will use the PGF in order to create low-level drawing macros which
  will be compiled by \texttt{xelatex}. We will also cover a generation of
  figures and legends separately which in some situations is beneficial. 
\end{abstract}

\section{Introduction}
This document presents the results of PGF output and \LaTeX symbiosis. It also
gives guidelines on how to achieve it. It is divided as follows.

\subsection{Prerequisites}
I tried to keep the document minimal so that you should not need many packages
in order to run the example. These are the essential bits
\begin{enumerate}
  \item \texttt{python2-matplotlib}
\end{enumerate}

\section{Example figures}

Several examples are presented here. On figure~\ref{fig:latex} we are simply
looking at the capability of plotting via pgf. Figure~\ref{fig:several} we are
mainly looking at using legend as as separate graph that is only tied to the
actual plots at the level of the document. This approach is beneficial if you
are plotting many figures of the same type and the legend would have been
repeated on each graph. Also, having legend outside of the plot makes the plot
clearer and there is nothing to obscure the results.


\begin{figure}[b] 
  \centering
  %note that we're using \input!
  \input{graphics/bessel.pgf} 
  \\
  \caption{I am printing word "abscissa" just to prove that the fonts are
  matched. We're also admiring the \LaTeX\ equations on the graph.}
  \label{fig:latex} 
\end{figure}
\begin{figure}[b] 
  \centering
  \input{graphics/c40.pgf} 
  \input{graphics/c60.pgf} 
  \input{graphics/c80.pgf} 
  \\
  \input{graphics/legend.pgf} 
  \caption{Several figures and a legend. Note the convenience of having the
  legend in the actual document. It doesn't have to be assigned to any subplot.}
  \label{fig:several} 
\end{figure}

\section{Future work}
\begin{enumerate}
  \item Include an automated build system like Make or CMake.
\end{enumerate}
\end{document}
